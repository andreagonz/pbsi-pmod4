\documentclass[12pt]{article}
\usepackage[left=2cm,right=2cm,top=2cm,bottom=2cm,letterpaper]{geometry}
\usepackage{lmodern}
\usepackage[T1]{fontenc}
\usepackage[utf8]{inputenc}
% \usepackage[spanish,activeacute]{babel}
\usepackage{hyperref}
\usepackage{graphicx}
\graphicspath{{media/}}
\usepackage{float}
\usepackage{caption}
% \usepackage[toc]{multitoc}
% \setcounter{tocdepth}{2}
% automata
\usepackage{tikz}
\usepackage{pmboxdraw} 
\usepackage{fancyvrb}

% \begin{thebibliography}{99}
% 
% \bibitem{uno}
%   Silberschatz, A. \& Baer, P. \& Gagne, G. (2013).
%   \emph{Operating System Concepts} (9a ed). 
%   Estados Unidos: Wiley.
%   
% \end{thebibliography}
\title{Plan de Becarios en Seguridad Informática\\Proyecto Módulo 4 - Manual de Usuario}
\author{Rafel Diez Gutierrez González\\Andrea Itzel González Vargas}
% \date{ALGO}
\setlength{\parindent}{0em}

\begin{document}
\maketitle

El presente proyecto realiza un \textit{scraping} de los buscadores web Bing, Google, DuckDuckGo y Board Reader, con la opcion de hacer busquedas en Pastebin. El lenguaje de programacion utilizado es \textsf{Python 3}.

\section{Instalacion de dependencias}
Para hacer la instalacion de las dependencias se debe de correr el script \texttt{scrappy/dependencias.sh}, tambien existe la posibilidad de ejecutar el comando
\begin{verbatim}
  $ python3 scrappy/setup.py install
\end{verbatim}
lo cual iniciara la instalacion de dependencias.

\section{Ejecucion}
Para ejecutar el programa se debe de utilizar el siguiente comando:

\begin{verbatim}
  $ python3 scrappy.py {<busqueda> [opciones] | [busqueda] {f --filetype | 
           s --site | h --help | p --ip | u --inurl | m --mail} [opciones]
\end{verbatim}

Donde las opciones tienen el siguiente significado:
\begin{verbatim}
  -h, --help            Muestra mensajes de ayuda
  -v, --verbose         Indica que se utilice el modo verboso.
  -n NUM_RES, --num-res=NUM_RES
                        Numero de resultados por busqueda
  -b BUSCADORES, --buscadores=BUSCADORES
                        Se especifica el buscador a utilizar
  -N, --no-params       Excluye los parametros GET, haciendo unica cada
                        busqueda
  -r, --regex           Se pueden usar Expresiones Regulares
  -d, --domains         Indica que se impriman unicamente los dominios en los
                        reportes
  -U USER_AGENTS, --user-agents=USER_AGENTS
                        Archivo con agentes de usuario a utilizar, separados
                        por un salto de linea
  -P PROXIES, --proxies=PROXIES
                        Archivo con proxies a utilizar, separados por un salto
                        de linea
  -F FORMATO, --formato=FORMATO
                        Especifica el formato de salida
  -i INTERVALO, --intervalo=INTERVALO
                        Se especifica el intervalo de tiempo por busqueda
  -p IP, --ip=IP        Se especifica la(s) IP(s) de busqueda, separadas por
                        una coma
  -m MAIL, --mail=MAIL  Busca correos electronicos en los dominios
                        especificados, separados por una coma
  -f FILETYPE, --filetype=FILETYPE
                        Se busca por los tipos de archivo especificados,
                        separados por una coma
  -s SITE, --site=SITE  Se busca por los sitios web especificados, separados
                        por una coma
  -e EXCLUDE, --exclude=EXCLUDE
                        Se excluyen los resultados que contengan las palabras
                        indicadas, separadas por una coma
  -w EXACT_WORD, --exact-word=EXACT_WORD
                        Se buscan las palabras indicadas de manera exacta,
                        cada una va separada por una coma
  -I INCLUDE, --include=INCLUDE
                        Se incluyen los resultados que contengan esa palabra
  -u INURL, --inurl=INURL
                        Se buscan las palabras dentro de la url, separadas por
                        comas
  -o OUTPUT, --output=OUTPUT
                        Nombre de los archivos de reporte.
  -t, --tor             Indica que se haga uso de tor para hacer las
                        peticiones (junto con otros proxies, si se utiliza la
                        opcion -P).
\end{verbatim}

El parametro \texttt{<busqueda>} es la busqueda principal a ser realizada, si esta tiene la forma \texttt{palabra1 + palabra2}, se indica que se haga la busqueda independiente del termino \texttt{palabra1} y el termino \texttt{palabra2}, es decir, el simbolo \texttt{+} es equivalente a un \texttt{OR}. \\

En el caso de utilizar la opcion \texttt{-r}, el parametro \texttt{<busqueda>} sera tomado como una expresion regular, cuyas expansiones comprenderan los terminos a ser buscados independientemente.

\section{Ejemplo de ejecucion}

Al ejecutar el siguiente comando se hace la busqueda del termino \texttt{gatos}, utilizando los buscadores Bing y Google. Unicamente se obtienen busquedas de archivos PDF en el dominio \texttt{unam.mx}, obteniendose aproximadamente 15 resultados por busqueda. Las peticiones se realizan anonimamente a traves de Tor. El reporte de resultados sera generado en tres formatos, \texttt{xml}, \texttt{html} y \texttt{txt}, los cuales tendran el nombre \texttt{busquedas.[html|xml|txt]}. Al estarse utilizando el modo verboso, se mostraran mensajes durante la ejecucion del programa.
\begin{verbatim}
  $ python3 scrappy.py -f pdf -b bing,google "gatos" -F xml,html,txt \
        -s unam.mx -n 15 -t -o busquedas -v
\end{verbatim}

\end{document}
